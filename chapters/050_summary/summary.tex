\chapter{Conclusions}
\label{chap:summary}

%\chapterquote{}{}

A measurement of the \PZ invisible width is performed with
$\SI{35.9}{fb^{-1}}$ of data collected by the CMS experiment during the 2016
proton-proton collisions at $\sqrt{s}=\SI{13}{TeV}$. The invisible width is
measured from the \metplusjets signature left in the detector as the \PZ boson
decays into neutrinos which pass through undetected. These are compared to the
muon and electron (collectively $\ell$) reference decays of the \PZ boson by
collecting events with the \diellplusjets signature. The selection placed on
these events is similar to the \metplusjets region to restrict to similar
kinematics. In particular, the \recoil is used in place of the missing
transverse energy as a proxy for the boson \pt in \IVj events. The modelling
of this variable, as a proxy for the boson \pt, is validated in data by
measuring its scale and resolution in events with a fully reconstructed
dilepton for the boson \pt. This parameter is further used in the \ellplusjets
control regions, where a data-driven estimate of the large \IWlvj background
in the \metplusjets region is performed through the transfer factor approach.
Other backgrounds are significantly smaller and estimated from MC, apart from
the QCD multijet process. A QCD multijet enriched sideband is collected by
inverting the selection on \mindphi. Within this region, a correction factor
for the QCD process is determined by extrapolation, with conservatively large
uncertainties on the small background. In addition to these background
estimations, the \IDYll process is divided into the \IZll and \Igstarll
constituents to determine the contributions from each subprocess and their
interference.

The invisible width is extracted with a simultaneous likelihood fit between
the \metplusjets, \ellplusjets and \diellplusjets regions. The parameter of
interest is extracted as a scaling parameter on the \IZvvj process, relative to \IZllj.
This parameter is translated into an invisible width measurement of
%
\begin{equation}
    \Gamma_{\mathrm{inv}} = 512 \pm 3\ (\mathrm{stat.})\ \pm 16\ (\mathrm{syst.})\ \mathrm{MeV}\ ,
\end{equation}
%
consistent with the SM and comparable to the LEP combined measurement. The measurement presented is systematically
dominated, with the limiting sources from the jet energy scale, muon
identification and trigger efficiency uncertainties.

The limiting systematic uncertainties imply that this measurements will not
show significant gains from a larger dataset from the continual running of the
LHC at increased luminosities. However, similar precision-based measurements
are crucial tests of the SM, especially with the lack of BSM physics at the LHC. 
These include furthering our understanding of the Higgs boson since its
discovery at the LHC, as well as probing the electroweak sector with precision
measurements.