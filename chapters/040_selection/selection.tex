\chapter{Event Selection and Categorisation}
\label{chap:analysis}

%\chapterquote{}{}
% \cite{Alwall:2014hca}.

\section{Introduction}

The measurement of the \IZinv width requires the collection, through event tagging of a jet, of \IZvvj signal events with the key signature of jets recoiling against an invisible system, also known as \metplusjets. This process is compared and contrasted to the similar \IDYllj reference process to extract a precise measurement of the invisible width. Additional samples of \IWlvj provide a data-driven estimate of large backgrounds with a similar signature to the signal. A combination of the large QCD multijet cross-section and inherent jet misreconstruction leads to additional backgrounds and the need of a sideband\footnote{A sideband is an independent set of events to a region obtained by inverting or somehow altering a requirement which defined the original region.} enriched with QCD multijet events for a data-driven estimate of this background.

This chapters details the process of data collection and reduction to target specific processes in various categories required for this analysis. Included is the detail of particle identification refinement from the PF algorithm results for an improved signal acceptance whilst maintaining a high background rejection.

\section{Samples}

\subsection{Data collection}

The data collection of physical proton-proton collections corresponds to retrieving the relevant data stored by a specific set of \SWT requirements, known as \SWT paths. The dataset considered was collected during 2016 with proton-proton collisions at a centre-of-mass energy of ${\SI{13}{TeV}}$.  Issues during detector live-time, such as problems with a particular subdetector or beam conditions, requires the masking of known problematic collisions resulting in a dataset with an integrated luminosity of ${\SI{35.9}{fb^{-1}}}$.

The relevant process can be classified into events with missing transverse energy or more than one lepton. Events with missing transverse energy are selected by the \SWT paths requiring two conditions or a third to be satisfied. The first is
%
\begin{equation}\label{eq:hlt_met_trig_selection1}
    \ptmissnomu = \bigg| -\sum_{i\not\in\mathrm{muons}} \vec{p}_{\mathrm{T}} \bigg| \ge \ptmissthresh\ ,
\end{equation}
%
where the sum is over all PF candidates apart from muons and \ptmissthresh is a particular threshold. The muons act as invisible particles for \ptmissnomu hence collects \IZvv, \IWmv and \IDYmm events. The second condition is
%
\begin{equation}
    p_{\mathrm{T,had}}^{\mathrm{miss}} = \bigg| -\sum_{i\in\mathrm{jets}} \vec{p}_{\mathrm{T}} \bigg| \ge \ptmissthresh\ ,
\end{equation}
%
summed over all PF jets passing the tight jet identification selection with ${\pt>\SI{20}{GeV}}$ and no muon candidates. This selection is similar to Eq.~\ref{eq:hlt_met_trig_selection1}, however, the combination of both maintains a trigger rate for lower \ptmissthresh  thresholds by removing events with inconsistent visible (excluding muons) and hadronic recoils, primarily due to misreconstructed soft (${\pt<\SI{20}{GeV}}$) hadronic activity. The threshold \ptmissthresh varies from {$90$--$\SI{120}{GeV}$} depending on the \LHC beam conditions to avoid prescaling. The third condition recovers small inefficiencies by applying the selection
%
\begin{equation}
    \ptmiss = \bigg| -\sum_{i\in\mathrm{PF}} \vec{p}_{\mathrm{T}} \bigg| \ge \SI{170}{GeV}\ ,
\end{equation}
%
summed over all PF candidates, with various beam halo or \HCAL noise cleaning requirements (discussed in Sec.~\ref{sec:baseline-selection}) to maintain this threshold as beam conditions change.

A complementary trigger path collects events with at least one muon rather than the momentum imbalance of an event. Events pass the trigger requirement if there exists a global or tracker muon with ${\pt>\SI{24}{GeV}}$ which is isolated from other activity (total energy within a cone of $\Delta R=0.3$ is below $9\%$ relative to the muon \pt). To collect events with at least one electron an alternative trigger path is used with the requirement of at least one GSF electron passing a tight identification selection and a ${\pt>\SI{27}{GeV}}$. Both lepton trigger thresholds are driven by a goal to maintain data collection efficiency and control trigger rates.


\subsection{Simulation}

A vast array of simulated proton-proton collisions attempt to fully replicate the events selected as signal candidates, including backgrounds with similar experimental signatures to the signal. Vector boson production; namely \IZvv, \IWlv, \IDYll and prompt photon; in association with jets, \IVj, are produced with the matrix element (ME) event generator \MADGRAPH \cite{Frederix:2018nkq} with restricted boson decays and Feynman diagrams up to $\mathcal{O}(\alpha^2\alpha_s^2)$, hereby known as next-to-leading order (NLO) in the strong coupling constant. The production is split into boson \pt regions to populate the low cross-section high-\pt regions. These generated events are coupled to the \PYTHIA package \cite{Sjostrand:2014zea} to mediate the parton showering (PS), hadronisation and particle decays with the \CUET underlying event description \cite{Khachatryan:2015pea}. Double-counting of jet multiplicity final states with the ME and PS is removed by the FXFX \cite{Frederix:2012ps} matching and merging technique. Another set of processes similar to \IDYll are generated where, in turn, the \PZ and \Pgstar interactions are turned off to model the \PZ, \Pgstar and interference terms of the \IDYll process. Furthermore, the processes involving a \Pgstar have an invariant mass cut-off below ${\SI{50}{GeV}}$ to avoid the large cross-section away from the \PZ peak.

Minor background processes include the pure QCD interactions between partons forming a multijet final state with jet misreconstruction leading to spurious \ptmiss, faking the signal signature. The QCD multijet process is produced at leading order (LO) accuracy in the strong coupling constant by the \PYTHIA package, split by the interaction scale $Q^2$ to populate the low cross-section regions. Top production leads to real \ptmiss with zero, one or two leptons in the final state with the pair production, \Ittj, produced using a similar technique as the \IVj processes, albeit inclusive in \pt, and single top production --- via the $s$-channel, $t$-channel or associated \PW production --- are produced at NLO by a various combination of the \MADGRAPH, \MADSPIN \cite{Artoisenet:2012st}, \PYTHIA and \POWHEG \cite{Alioli:2010xd,Alioli:2009je,Re:2010bp} packages. Diboson processes (combination of \PZ, \PW and prompt \Pgamma from the hard scatter) may have a similar signature to the signal for particular decays, or particles beyond the detector's acceptance or misidentified. Combinations of \MADGRAPH, \MADSPIN, \PYTHIA and \POWHEG are used to generate the diboson process at NLO. Finally, vector boson scattering (VBS) involves the production of a vector boson (\PW or \PZ) from $\PV\PV$ scattering, in association with two jets, hence has a signature similar to the \IVj processes. The VBS sample is produced at LO accuracy by the \MADGRAPH package coupled to \PYTHIA, with the boson decay restricted to charged leptons or neutrinos. An overlap between the phase-space of the diboson and VBS samples is removed by an ad hoc requirement of one generated boson from the hard scatter for VBS.

All samples are generated with \NNPDF \cite{Ball:2014uwa} LO and NLO parton distribution functions, where the LO and NLO generators are used. After the relevant process has been generated the full detector response to such an interaction, along with pileup vertices, is simulated with the \GEANT \cite{AGOSTINELLI2003250} package.


\section{Physics objects}

Analysis specific physics objects are derived from PF candidates to refine and target the selection towards the key signature. These include muons; electrons; photons; jets with tagging as hadronic $\tau$-lepton decays or $b$-hadrons; and the global event descriptor $\ptmiss$, redefined in terms of the recoil parameter.

\subsection{Muons}

The analysis definition of a muon is split into two categories: veto and selection muons. The selection defining these categories is shown in Tab.~\ref{tab:muon-selection}. Muons from in flight hadron decays are suppressed by a requirement on the particle-based isolation $I_{\Delta\beta}$ with a pileup mitigation term, known as the $\Delta\beta$ correction
%
\begin{equation}
    I_{\Delta\beta} = \frac{1}{\pt}\left| \sum_{i\in\mathrm{ch}}\vec{p}_{\mathrm{T},i} + \max\left( 0, \sum_{i\in\mathrm{nh}}\vec{p}_{\mathrm{T},i} + \sum_{i\in\gamma}\vec{p}_{\mathrm{T},i} - \frac{1}{2}\sum_{i\in\mathrm{PU\ ch}}\vec{p}_{\mathrm{T},i} \right)\right|\ ,
\end{equation}
%
for a muon with transverse momentum \pt; charged hadrons (ch), neutral hadrons (nh) and photons (\Pgamma) from the primary vertex in a cone of $\Delta R=0.4$ around the muon; and charged hadrons from pileup vertices (PU ch) also within this cone. This correction mitigates the contribution from neutral hadrons from pileup vertices by subtracting the momentum of charged hadrons from these vertices, with a factor of a half to account for the average charged-to-neutral particle ratio in the isospin limit. Additional selection is placed on the quality of selection muon tracks, including the impact parameter with the primary vertex split by the transverse, $d_{xy}$, and longitudinal, $d_{z}$, components.

\begin{table}[htb]
    \centering
    \begin{tabular}{lcc}
        \hline\hline
        & Veto & Selection \\
        \hline
        Muon seed & Global or tracker & Global \\
        PF muon & True & True \\
        Track fit normalised $\chi^2$ & - & $<10$ \\
        Number of matched stations & - & $>1$ \\
        Number of muon chamber hits & - & $>0$ \\
        $d_{xy}$ (cm) & - & $<0.2$ \\
        $d_{z}$ (cm) & - & $<0.5$ \\
        Pixel hits & - & $>0$ \\
        Tracker layer hits & - & $>5$ \\
        $I_{\Delta\beta}$ & $<0.25$ & $<0.15$ \\
        \pt (GeV) & $>10$ & $>30$ \\
        \aeta & $<2.5$ & $<2.4$ \\
        \hline\hline
    \end{tabular}
    \caption[Requirements placed on muons.]{
        Requirements for muons passing the veto and selection identification.  Most track quality criteria are omitted for veto muons to retain a high efficiency at the loss of fake muon rejection \cite{CMS-DP-2017-007}.
    }
    \label{tab:muon-selection}
\end{table}

%The veto criteria are highly efficient at selecting prompt muons and muons from heavy and light quark decays, while the selection criteria 

\subsection{Electrons}

Similar to muons, the electrons are classified as veto or selection based on criteria detailed in Tab.~\ref{tab:electron-selection}. The first requirements primarily reduce fake electrons from jets with criteria on $\sieiefbf$, the extent of the shower along the \ECAL crystals in the $\eta$ direction within a $5\times 5$ crystal cluster, typically smaller for electrons and photons than hadronic showers; and $E_{\mathrm{\HCAL}}/E_{\mathrm{\ECAL}}$, the ratio of the energy deposited in \HCAL towers located behind the \ECAL cluster to the \ECAL cluster energy, significantly smaller for electron and photons compared to jets. Subsequent requirements check the consistency of the GSF track with the \ECAL cluster, namely $|\Delta\eta_{\mathrm{In}}|$, the difference in $\eta$ of the energy-weighted supercluster centre and the GSF track extrapolated to the \ECAL; $|\Delta\phi_{\mathrm{In}}|$, similarly defined in the $\phi$ direction; $|(1/E_{\mathrm{\ECAL}} - 1/p_{\mathrm{T,track}})|$; and the impact parameters $d_{xy}$ and $d_{z}$. The conversion veto rejects electrons with tracks consistent with photon conversions identified by the conversion finder. The isolation, $I_{\mathrm{EA}}$, requirement on electrons reduces contamination from in flight hadron decays, with a pileup mitigation technique known as effective area:
%
\begin{equation}
    I_{\mathrm{EA}} = \frac{1}{\pt} \left| \sum_{i\in\mathrm{ch}} \vec{p}_{\mathrm{T},i} + \max\left(0, \sum_{i\in\mathrm{nh}}\vec{p}_{\mathrm{T},i} + \sum_{i\in\Pgamma}\vec{p}_{\mathrm{T},i} - \rho A_{\mathrm{eff}}\right) \right|\ ,
\end{equation}
%
which is similar to $I_{\Delta\beta}$, however, the subtracted quantity is the product of the energy density $\rho$ of the event with the ($\eta$-dependent) effective area of the electron, a pre-computed average over multiple events.

\begin{table}[htb]
    \centering
    \begin{tabular}{lcccc}
        \hline\hline
        & \multicolumn{2}{c}{Veto} & \multicolumn{2}{c}{Selection} \\
        & Barrel & Endcaps & Barrel & Endcaps \\
        \hline
        GSF track & \multicolumn{4}{c}{True} \\
        \sieiefbf & 0.011 & 0.0314 & 0.00998 & 0.0292 \\
        $E_{\mathrm{\HCAL}}/E_{\mathrm{\ECAL}}$ & 0.298 & 0.101 & 0.0414 & 0.0641 \\
        $|\Delta\eta_{\mathrm{In}}|$ & 0.00477 & 0.00868 & 0.00308 & 0.00605 \\
        $|\Delta\phi_{\mathrm{In}}|$ & 0.222 & 0.213 & 0.0816 & 0.0394 \\
        $|(1/E_{\mathrm{\ECAL}} - 1/p_{\mathrm{T,trk}})|$ (1/GeV) & 0.241 & 0.14 & 0.0129 & 0.126 \\
        $d_{xy}$ (cm) & 0.05 & 0.1 & 0.05 & 0.1 \\
        $d_{z}$ (cm) & 0.1 & 0.2 & 0.1 & 0.2 \\
        Missing tracker hits & \multicolumn{4}{c}{2} \\
        Conversion veto & \multicolumn{4}{c}{True} \\
        $I_{\mathrm{EA}}$ & 0.0994 & 0.107 & 0.0588 & 0.0571 \\
        \pt (GeV) & \multicolumn{2}{c}{$>10$} & \multicolumn{2}{c}{$>30$} \\ 
        \hline\hline
    \end{tabular}
    \caption[Criteria to select and veto electrons.]{
        Criteria for the electrons identified as veto or selection for the analysis. All parameters must be less than the value unless stated otherwise. The criteria are split by barrel and endcap regions to retain similar electron identification efficiency in both \cite{CMS-DP-2017-004}.
    }
    \label{tab:electron-selection}
\end{table}

\subsection{Photons}

The selection for photons is closely aligned with electrons, although with a single veto category and without tracker based requirements and an explicit rejection of candidates linked to tracks. Furthermore, the isolated requirements are split by particle candidates: charged hadrons, neutral hadrons and photons; with a \pt-dependence to remain efficient across a broad range of momenta. The selection is detailed in Tab.~\ref{tab:photon-selection}.

\begin{table}[htb]
    \centering
    \begin{tabular}{lcc}
        \hline\hline
        & Barrel & Endcaps \\
        \hline
        Linked track & False & False \\
        \sieiefbf & 0.01031 & 0.03013 \\
        $E_{\mathrm{\HCAL}}/E_{\mathrm{\ECAL}}$ & 0.0597 & 0.0481 \\
        $I_{\mathrm{ch}}$ & $1.295/\pt$ & $1.011/\pt$ \\
        $I_{\mathrm{nh}}$ & $10.910/\pt + 0.00148 + \num{1.7e-5}\pt$ & $5.931/\pt + 0.0163 + \num{1.4e-5}\pt$ \\
        $I_{\Pgamma}$ & $3.630/\pt + 0.0047$ & $6.641/\pt + 0.0034$ \\
        \pt (GeV) & \multicolumn{2}{c}{$>25$} \\
        \hline\hline
    \end{tabular}
    \caption[Photon requirements to veto events.]{
        Photon requirements for the veto collection with all parameters lower than the values in the table unless stated otherwise. The criteria are split by barrel and endcap regions as with electrons \cite{CMS-DP-2017-004}. Only veto photons are defined for this analysis.
    }
    \label{tab:photon-selection}
\end{table}

\subsection{Jets}

Requirements on jets, detailed in Tab.~\ref{tab:jet-selection} depend on the region which they fall into with a veto and a selection category defined below $\aeta=2.4$ and inclusive in $\eta$, respectively. Pileup suppression is implemented by removing charged hadrons associated to pileup vertices from the jet cone. Leptons or photons within their respective veto collection which reside within a cone of a jet results in the removal of the jet from the event.

\begin{table}[htb]
    \centering
    \begin{tabular}{lccc}
        \hline\hline
        & Central & Endcaps & Forward \\
        \hline
        Charged hadron energy fraction & $>0$ & - & - \\
        Charged EM energy fraction & $<0.99$ & - & - \\
        Charged particle multiplicity & $>0$ & - & - \\
        Constituent multiplicity & $>1$ & $>1$ & - \\
        Neutral hadron energy fraction & $<0.99$ & $<0.99$ & - \\
        Neutral EM energy fraction & $<0.99$ & $<0.99$ & $<0.9$ \\
        Neutral particle multiplicity & - & - & $>10$ \\
        \pt (GeV) & \multicolumn{3}{c}{$>40$} \\
        \hline\hline
    \end{tabular}
    \caption[Jet selection requirements.]{
        Jet selection based on position: central ($\aeta<2.4$), endcaps ($2.4<\aeta<3$) and forward ($\aeta>3$). Distinction between charged and neutral particles is lost beyond the tracker acceptance of $\aeta<2.4$ and forward jets are required to have a significant neutral particle number due to their high radiation environment.
    }
    \label{tab:jet-selection}
\end{table}

Additional selection applied to jet structure allows the tagging of hadronic $\tau$-lepton decays or jets originating from $b$-hadrons.

\subsubsection{Hadronic $\tau$-lepton tagging}

The $\tau$-lepton decays into hadrons and a neutrino, labelled as \Ptauh, $64.8\%$ of the time, with the remaining decays into an electron or muon and neutrinos. The production of a \PW boson and its subsequent decay \IWthv leads to the signal signature of \metplusjets, therefore, tagging and vetoing events with a \Ptauh-lepton is vital. The tagging is accomplished with the hadrons-plus-strips (HPS) \cite{Khachatryan:2062435} algorithm.

The branching fractions of the $\tau$-lepton are shown in Tab.~\ref{tab:tau-decays} with a significant \Ptauh branching fraction into one or three charged hadrons and up to two \Ppizero, detected from their decay into two photons which typically convert into $e^+e^-$ pairs within the tracker. Therefore, the HPS algorithm is seeded by jets with ${\pt>\SI{20}{GeV}}$ and ${\aeta<2.3}$ and matches charged hadrons to $\eta$-$\phi$ strips of photon and electron constituents. The strips are generated by an iterative procedure starting from the highest \pt electron or photon candidates and merging the next highest \pt candidate which lies in an $\eta\times\phi$ window of $0.05\times 0.20$, enlarged in $\phi$ to accommodate electron bending, around the strip centre. After merging the strip centre is updated from an energy-weighted average of the merged candidates and the iteration proceeds until no further candidates are available. Upon generating all strips, only those with a \pt greater than $\SI{2.5}{GeV}$ are kept as \Ppizero candidates.

\begin{table}[htb]
    \centering
    \begin{tabular}{lr}
        \hline\hline
        Decay mode & $\mathcal{B}$ [\%] \\
        \hline
        $\tau^-\ra\ell\bar{\nu}_{\ell}\nu_{\tau}$ & $35.2$ \\
        \hline
        $\tau^-\ra h^-\nu_{\tau}$ & $11.5$ \\
        $\tau^-\ra h^-\Ppizero\nu_{\tau}$ & $25.9$ \\
        $\tau^-\ra h^-\Ppizero\Ppizero\nu_{\tau}$ & $9.5$ \\
        $\tau^-\ra h^-h^+h^-\nu_{\tau}$ & $9.8$ \\
        $\tau^-\ra h^-h^+h^-\Ppizero\nu_{\tau}$ & $4.8$ \\
        Remaining hadronic decays & $3.3$ \\
        \hline
        All hadronic decays & $64.8$ \\
        \hline\hline
    \end{tabular}
    \caption[Branching fractions of the $\tau$-lepton decay modes.]{
        Branching fractions $\mathcal{B}$ for $\tau$-lepton decay modes without particular distinction between charged hadron, $h^{\pm}$, type and highlighting the decay modes targeted by the HPS algorithm. The two hadron final state is typically mediated by $\rho(770)$ resonance and the three hadron final state by $a_1(1260)$. The table is adjusted \cite{Khachatryan:2062435} with approximate values taken from the particle data group (PDG) \cite{PhysRevD.98.030001}.
    }
    \label{tab:tau-decays}
\end{table}

The \Ptauh-lepton candidates are formed by matching strips to charged particle candidates within the seeding jet and consistent with the vertex nearest to the leading track: nearest approach to the vertex of at most $\SI{0.4}{cm}$ in $z$ and $\SI{0.03}{cm}$ in the $x$-$y$ plane to remove spurious tracks and mitigate pileup effects without a significant loss in efficiency. All \Ptauh-lepton candidates with one or three charged hadrons and up to two strips, within a cone $\Delta R=(\SI{3}{GeV}/\pt)$, are classified into a particular decay mode if the invariant mass of all the hadrons is consistent with the $\tau$-lepton mass, with some margin to account for the neutrino. The \Ptauh-lepton is assigned the decay with the highest \pt from the remaining modes for a unique classification.

Misidentification of quark or gluon jets as a \Ptauh-lepton may result in a loss in signal efficiency for \IZvvj when vetoing \Ptauh events. These are recovered by discriminating \Ptauh-leptons and quark or gluon jets with an isolation-based BDT. The input features vary upon the classified decay mode and include: the lead track $d_{xy}$ and its significance, the distance between the $\tau$-lepton production and decay vertices and its significance, charged and neutral particle isolation sums and the $\Delta\beta$ correction value, an integer encoding the decay modes, a success flag upon finding a decay vertex for the $\tau$-lepton, along with the \pt and $\eta$ of the $\tau$-lepton. This BDT is trained and validated on simulated events with a selection on the discrimination variable defining various signal efficiency and background rejection, adjusted by the $\tau$-lepton \pt, working points. A very loose working point with a high efficiency is important for this analysis for an efficient rejection of \Ptauh-lepton backgrounds. Further discrimination on electron or muon misidentification as the charged hadron candidate are omitted in favour of higher selection efficiencies.

\subsubsection{$b$-jet tagging}

The jets associated to the ${\PZ(\ra\nu\bar{\nu})}$ production are typically quark or gluon jets with processes involving neutrinos in the final state along with $b$-jets result in a similar signature, such as top-quark production. Therefore, discrimination of $b$-jets from quark or gluon jets and the subsequent veto of events with $b$-jets is vital for this analysis.

The $b$-jet tagging is performed on analysis-level jets by applying criteria on the jet's constituent tracks and vertices to identify $b$-hadron decays with displacements of about $\SI{1}{cm}$, hence only valid for jets within the tracker's acceptance ${\aeta<2.4}$. First, a set of secondary vertices are found by an inclusive vertex finding (IVF) algorithm from all tracks within an event with ${\pt>\SI{0.8}{GeV}}$ and a longitudinal impact parameter less than ${\SI{3}{mm}}$. The vertex finder is initialised by track seeds defined by a 3-dimensional impact parameter of at least ${\SI{50}{\micro m}}$ and a 2-dimensional impact parameter significance above $1.2$. From these seed tracks the algorithm follows \cite{Sirunyan:2017ezt}:
\begin{itemize}
    \item \textbf{Track clustering:} grouping the seed track with compatible tracks based on separation and opening angle.
    \item \textbf{Secondary vertex fitting and cleaning:} fitting the position of the secondary vertex from the clustered tracks with the adaptive vertex fitter (discussed in Sec.~\ref{sec:vertex-reco}). Identified vertices must have 2 and 3-dimensional flight distances consistent with $b$ or $c$-hadron decays and vertices sharing $70\%$ of tracks with another nearby vertex are removed.
    \item \textbf{Track arbitration:} tracks associated with both primary and secondary vertices are removed if they are more compatible with the primary vertex.
    \item \textbf{Refitting and cleaning:} positions of the secondary vertices are refitted and duplicate vertices are removed.
\end{itemize}
Resulting set of secondary vertices are associated to jets if the angular distance between the jet axis and the vertex is within ${\Delta R=0.3}$. The jets seeding the b-tag fall into three independent categories based on the number of secondary vertices and tracks: at least one secondary vertex, no secondary vertex but at least two tracks, and the remaining set of candidates.  An artificial neural network\footnote{An artificial neural network is a form of machine-learning, inspired by biological neural networks, with inter-connected nodes typically with a non-linear function output of a weighted sum of its inputs. The weights are adjusted in the learning process to optimise a loss function of the output nodes and the truth category or value of a set of inputs.} is trained on each category with simulated events to discriminate $b$, $c$ and light ($u$, $d$, $s$ or gluon) jets. The artificial neural network is fed a large set of input features based on the jet and its constituent secondary vertices and tracks with parameters based on distance, invariant mass and multiplicities along with the jet \pt and $\eta$ \cite{Sirunyan:2017ezt}. Subsequently, a jet is $b$-tagged if the resulting discriminating variable is above a threshold defined by a medium working point where the misidentification probability is about $1\%$, resulting in a $b$-jet tagging efficiency of about $63\%$ \cite{Sirunyan:2017ezt}.

\section{Missing transverse momentum}

The transverse momentum imbalance of an event is reconstructed from the negative sum of transverse momenta of all PF candidates. This is used as an estimated of the vector boson \pt in \IZvvj events, where it cannot be determined directly. Therefore, in the collection of events targeting vector boson decays with partial or fully reconstructed decay products a proxy is used for the vector boson \vecpt, known as the recoil $\vec{\recoil}$, defined as
%
\begin{equation}
    \vec{\recoil} = \vec{p}_{\mathrm{T}}^{\mathrm{miss}} + \sum_{i\in\ell} \vec{p}_{\mathrm{T},i}\ ,
\end{equation}
%
where $i$ sums over all electrons and muons passing the analysis selection. In effect, this parameter determines the missing transverse momentum of an event treating electrons and muons as neutrinos.


\section{Event categorisation}

A set of categories are defined and used throughout this analysis for various purposes including signal collection, background prediction, and validation.  First, a baseline selection is placed on all regions furthered by region-specific criteria to define independent categories. The baseline selection is driven by the desired event topology, trigger thresholds and rejection of spurious \ptmiss backgrounds.


\subsection{Baseline selection}\label{sec:baseline-selection}

The first set of requirements, known as the \ptmiss filters, remove scarce events which typically have spurious \ptmiss. All events without a reconstructed primary vertex within acceptable parameters of the beam spot are discarded since the offset of the origin biases the \ptmiss distribution and few simulated events predict this contribution. Although infrequently an issue, machine induced particles, specifically high energy muons, may interact with the calorimeters with a significant energy deposit leading to spurious \ptmiss. Such events are removed by searching for atypical calorimeter shower shapes matching to CSC detectors. Further noise in the \HCAL subdetectors readout system may gives rise to spurious \ptmiss, hence event with \HCAL readout channels consistent with significant noise leading to a biased \ptmiss are discarded \cite{CMS-DP-2016-061}. Also as a result of noise \ECAL crystals may be masked during reconstruction, however, this may lead to a significant loss of energy if, for example, a jet is incident upon the masked crystal leading to spurious \ptmiss. These events are removed by inspecting \HWT towers to determine if a significant amount of energy was lost. A final source of spurious \ptmiss is a result of high \pt, low resolution PF muons which lead to few unaccounted issues in the reconstruction, removed by inspecting candidate events in depth.

The next criteria target the \metplusjets topology with thresholds driven by the missing energy trigger paths. To remain efficient in collecting events with these trigger paths the selection ${\mathcal{U}=|\vec{\mathcal{U}}|>\SI{200}{GeV}}$ (discussed in Sec.~\ref{sec:met-trigger-efficiency}) is applied with the recoil in place of the \ptmiss to target similar topologies between events with missing energy from neutrinos and those with leptons in the final state. The topology is further targeted by requiring a lead jet ${\pt>\SI{200}{GeV}}$ within tracker acceptance (${\aeta<2.4}$), avoiding dijet-like events and resulting in the jet as a primary recoil against the vector boson. An additional requirement on the charged hadron energy fraction between {$0.1$--$0.95$} for this lead jet reduces spurious \ptmiss events arising from detector noise and machine-induced particles.

Furthermore, events are reject if an object passing the veto and not the selection requirements for a jet, $b$-tagged jet, or a photon is observed within the events. These vetoes primarily remove events with VBS, top production and \Igj processes.

The final baseline requirement is
%
\begin{equation}
    \frac{\big| \ptmiss - \ptmisscalo \big|}{\recoil} < 0.5\ ,
\end{equation}
%
which tests the compatibility of the particle-based \ptmiss and the muon-corrected calorimeter-based \ptmisscalo, relative to the recoil. This inconsistency is typically due to an issue in the reconstruction and does not significantly impact the signal.


\subsection{Categorisation}\label{sec:categorisation}

The event categorisation is shown in Tab.~\ref{tab:event-categorisation} with lepton selections and vetoes defining each region and possible requirements to enhance the signal. The \ellplusjets ($\ell=e,\mu$) regions have an accompanied transverse mass selection on the \PW boson, \mtell, approximated as
%
\begin{equation}
    \mtell = \sqrt{2 \mom_{\mathrm{T},\ell} \ptmiss \left( 1 - \cos(\Delta\phi)\right)}\ ,
\end{equation}
%
where $\mom_{\mathrm{T},\ell}$ is the transverse momentum of the leptons and $\Delta\phi$ is the opening angle between the lepton and the \vecptmiss vectors. The window placed on \mtell surrounds the \PW mass to enhance these regions in \IWj events and reduce combinatorial backgrounds. Similarly, the \diellplusjets regions regions are accompanied by an invariant mass \mellell window centred on the \PZ mass, significantly reducing the combinatorial background and contributions from \Igstarj. A remaining large contribution remains from the QCD multijet process whereby spurious \ptmiss arises from jet mismeasurement resulting in an overlap between the \ptmiss and a jet. To avoid significant signal rejection only the overlap between the leading four jets are considered:
%
\begin{equation}
    \mindphi = \min_{i\in\{j_0,j_1,j_2,j_3\}} \Delta\phi\left(\vecptmiss, \vec{\mom}_{\mathrm{T},i}\right)\ ,
\end{equation}
%
where $i$ represents the set of the leading four jets and $\Delta\phi(\vec{a},\vec{b})$ represents the opening angle between vectors $\vec{a}$ and $\vec{b}$. The selection on \mindphi is inverted to obtain a QCD multijet enriched sideband. The \eleplusjets region has an additional requirement placed on the \ptmiss to further avoid the QCD multijet process with jets misidentified as electrons.

\begin{table}[htb]
    \centering
    \begin{tabular}{ll}
        \hline\hline
        Category & Requirements \\
        \hline
        \metplusjets & No additional veto electrons, muons or \Ptauh-leptons \\
        \hline
        \muplusjets & A single selected muon \\
        & No additional muons, or veto electrons or \Ptauh-leptons \\
        & ${30 < \mtmu < \SI{125}{GeV}}$ \\
        \hline
        \dimuplusjets & Two selected and oppositely charged muons \\
        & No additional muons, or veto electrons or \Ptauh-leptons \\
        & ${71 < \mmumu < \SI{111}{GeV}}$ \\
        \hline
        \eleplusjets & A single selected electron \\
        & No additional electrons, or veto muons or \Ptauh-leptons \\
        & ${30 < \mte < \SI{125}{GeV}}$ \\
        & $\ptmiss>\SI{100}{GeV}$ \\
        \hline
        \dieleplusjets & Two selected and oppositely charged electrons \\
        & No additional electrons, or veto muons or \Ptauh-leptons \\
        & ${71 < \mee < \SI{111}{GeV}}$ \\
        \hline
        \tauplusjets & A single selected \Ptauh-lepton \\
        & No additional \Ptauh-leptons, or veto electrons or muons \\
        \hline\hline
    \end{tabular}
    \caption[Analysis event categorisation.]{
        Definitions of each analysis category with the addition of the baseline selection and ${\mindphi>0.5}$, and an additional set of mirroring categories with the same definition but in the \mindphi sideband, ${\mindphi<0.5}$. A description of the variables is provided in the main text.
    }
    \label{tab:event-categorisation}
\end{table}


\section{Summary}

The data samples relevant for this analysis was collected during 2016 with the \ptmiss, muon and electron-based triggers. The data is complemented by a vast array of simulated events, predominantly from \IVj, generated at NLO accuracy in the strong coupling constant. Minor background processes are produced with a variety of event generators at LO and NLO accuracy. The initiating parton in all generated events are sampled from parton distribution functions from the \NNPDF group at LO and NLO accuracy where applicable. Generated events are typically coupled to a package to mediated parton showering, hadronisation and particle decays. Subsequent, detector simulation and signal digitisation is performed with the \GEANT package, including an overlay of pileup vertices.

Analysis level physics objects are defined with a veto and selection working point. Leptons and photons are required to pass a pileup-corrected isolation requirement. Furthermore, leptons are ignored from the \ptmiss calculation to define the recoil of an event. Jets are subcategorised in tags as originating from a hadronic $\tau$-lepton or $b$-hadron decays. The objects are used in the categorisation of each event with a baseline selection to filter rare spurious \ptmiss events, apply trigger thresholds and a lead jet selection to target the topology of a primary jet recoiling against a vector boson. Vetoes are placed on events containing forward (${\eta>2.4}$) jets, $b$-jets and selected photons. The final baseline condition removes misreconstructed events with an inconsistent \ptmiss and \ptmisscalo. Subsequent categorisation falls into \metplusjets, \ellplusjets ($\ell=e,\mu$ or \Ptauh) or \diellplusjets regions, with mirroring regions in a QCD multijet enriched sideband. Mass windows are placed on the combinations of the leptons and the \ptmiss to enrich the signal in all applicable regions. The resulting regions are used throughout the analysis.
