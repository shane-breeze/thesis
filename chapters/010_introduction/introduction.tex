\chapter{Introduction}
\label{chap:introduction}

\pagenumbering{arabic}
%\chapterquote{}{}

The classification and description of fundamental particles and their interactions by the standard model (SM) of particle physics has been successfully confirmed by countless experiments. However, it lacks a candidate for dark matter \cite{Ade:2015xua,Corbelli:1999af}, a neutrino oscillation mechanism \cite{Fukuda:1998ah,Ahmad:2002jz,Eguchi:2002dm}, or incorporating the gravitational interaction. At the LHC, the SM gained further success upon the discovery of the Higgs boson \cite{Aad:1471031,Chatrchyan:1471016}. Although numerous direct searches for beyond the SM (BSM) have been performed, no clear signal of BSM physics has been observed. As these searches continue to probe the extremities of distributions for BSM physics, an alternative direction proceeds through precision measurements of the SM, as presented in this thesis.

The SM predicts that about $20\%$ \cite{PhysRevD.98.030001} of the time a \PZ boson decays into a neutrino-antineutrino pair. This decay constitutes the \PZ invisible width, a signal not discernible in collider-based experiments.  Instead, it is inferred from the missing momentum associated with the invisible decay of the \PZ as it recoils against a detectable system. A measurement of the \PZ invisible width is a key test of the SM and can be translated into constraints on the number of neutrino species coupling to the \PZ boson. Deviations from the SM expectation may point towards the existence of new particles beyond the SM which contribute to the \PZ invisible width. Therefore, a precise measurement of this quantity could reveal signs of new physics

The experiments at LEP measured the invisible width of the \PZ using two methods. The direct method measures the rate of an energetic photon recoiling against an invisible system attributed to the s-channel \PZ production and subsequent invisible decay with an associated initial state radiated (ISR) photon: $e^+e^-\ra \gamma Z \ra \gamma\nu\bar{\nu}$. The indirect method measures the total \PZ width, by studying the lineshape of the \PZ boson from $e^+e^-$ collisions with a centre of mass energy near the \PZ mass, and subtracts the partial widths to all visible final state. This method provides the most precise measurement, with a sensitivity to BSM particles. A summary of the measurements performed by the LEP experiments L3 \cite{Acciarri:1998vf}, OPAL \cite{Akers:1994vh} and ALEPH \cite{Buskulic:1993ke} for the direct method, a combination and the indirect measurement is shown in Tab.~\ref{tab:lep-zinv-width}.

\begin{table}[htb]
    \centering
    \begin{tabular}{cc}
         \hline\hline
         Experiment & $\Gamma_{\mathrm{inv}}$ (MeV) \\
         \hline
         L3 & $498\pm 12\pm 12$ \\
         OPAL & $539\pm 26\pm 17$ \\
         ALEPH & $450\pm 34\pm 34$ \\
         LEP combined & $503\pm 16$ \\
         \hline
         LEP indirect & $499.0\pm 1.5$ \\
         \hline
    \end{tabular}
    \caption{
        Summary of the measurements of the $\Gamma_{\mathrm{inv}}$ from the LEP experiments for the direct measurements (with the statistical and systematic uncertainties), its combination, and the indirect measurement.
    }
    \label{tab:lep-zinv-width}
\end{table}

To date, there is no published measurements of the \PZ invisible width at a hadron collider. However, a precise measurement at the LHC complements the LEP measurements through an independent test of the SM at a different energy scale, exposing contributions to the invisible width to TeV scale contributions. These additional contributions may come from couplings of the \PZ with dark matter particles, such as the lightest supersymmetric candidate, or a mixing between the neutrino and sterile neutrino states \cite{Carena:2003aj}, to which both the direct and indirect measurements are sensitive. However, the direct measurement is also sensitive to contributions to the invisible final state without coupling to the \PZ boson, such as a heavier mediator coupling to neutrinos or other exotic invisible particles, or a four fermion interaction vertex \cite{Carena:2003aj}.  Therefore, both measurements are valuable tests of the SM with the direct measurement a candidate for the LHC to provide a complementary measurement.

The experiment and analysis presented by the author is a direct measurement of the invisible width of the \PZ boson with the CMS detector using data from proton-proton collisions at ${\sqrt{s}=\SI{13}{TeV}}$ in 2016, corresponding to an integrated luminosity of $\SI{35.9}{fb^{-1}}$. The measurement targets the invisible decay of the \PZ bosons recoiling against a hadronic system. The rate of the invisible final state is compared to the reference signal of the partial widths to leptons $\Gamma(\PZ\ra\ell\ell)$, since these decays are kinematically similar. The invisible width is then given by
%
\begin{equation}\label{eq:zinv}
    \Gamma(\IZinv) = \frac{\sigma(\IZj)\mathcal{B}(\IZinv)}{\sigma(\IZj)\mathcal{B}(\IZll)}\Gamma(\IZll)\ ,
\end{equation}
%
where $\sigma(\IZj)$ is the cross section for the \IZj process, modified by either the invisible branching fraction $\mathcal{B}(\PZ\ra\mathrm{inv.})$ or dilepton branching fraction $\mathcal{B}(\IZll)$.
