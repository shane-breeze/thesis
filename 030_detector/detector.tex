\chapter{The \LHC and \CMS detector}
\label{chap:detector}

%\chapterquote{}{}

%------------------------------------------------------------------------------%
\section{Introduction}

% TODO: Careful of overlap with introduction chapter
The \LHC was built with the design goal of leading the energy frontier of high
energy physics. This led to the successful discovery of the Higgs bosons by the
\ATLAS \cite{Aad:1471031} and \CMS experiments \cite{Chatrchyan:1471016}.
Moreover, \BSM searches with these experiments continue to push stringent
limits on the possible parameter-space of theorised new physics scenarios. Such
discoveries and searches are only possible by the hugely complex detectors
and the \LHC accelerator complex, built and maintained by thousands of Engineers
and Physicists.

%------------------------------------------------------------------------------%
\section{The \LHC}

\subsection{Accelerator Complex}

Proton collisions at the \LHC start from the ionisation of hydrogen gas to
liberate the proton nuclei from their bound atomic states. The protons are
accelerated by the RF-cavity based accelerator \LINACTWO and injected into the
\PSBooster --- four superimposed synchrotron rings accelerating protons from
${\SI{50}{MeV}}$ to ${\SI{1.4}{GeV}}$ --- permitting the injection of more
protons into the \PS. The \PS is a synchrotron with a radius of
${\SI{72}{m}}$ bending the proton beam into a ring with energies up to
${\SI{25}{GeV}}$ before injection into the \SPS. Prior to the injection the
proton beam is bunched into discrete packets of protons about ${\SI{4}{ns}}$
long with an equal spacing of ${\SI{25}{ns}}$ required by the \LHC for a
discrete collision rate of ${\SI{40}{MHz}}$. The \SPS is another synchrotron
with the larger radius of ${\SI{6.9}{km}}$ for improved bending power required
for the acceleration of protons up to ${\SI{450}{GeV}}$ for extraction by the
\LHC. The whole accelerator complex allows the \LHC to accelerate protons
bunches at energies of ${\SI{6.5}{TeV}}$ with a bunch crossing of
${\SI{25}{ns}}$ and luminosities of the order of ${\SI{e34}{cm^{-2}s^{-1}}}$.

    \begin{itemize}
        \item \cite{Mobs:2197559}, \cite{Benedikt:823808}
    \end{itemize}

    \subsection{\LHC Main Ring}

    \begin{itemize}
        \item Goal: Higgs discovery and rare events. Require high luminosity at
            energy frontier.
        \item High intensity excluded proton-antiproton and a single vacuum chamber.
            Protons beams circulate in opposite directions within separate vacuum
            chambers. Common sections in the insertion to the detectors for
            collision.
        \item Beams circulate in (nominally) 2808 bunches with a spacing of
            ${\SI{25}{ns}}$.
        \item Twin bore magnets: two sets of coils and beam channels within the same
            structure and cryostat.
        \item Peak energies of ${\SI{7}{TeV}}$ per proton requires peak magnetic
            field of ${\SI{8.33}{T}}$. Superconducting magnets are used.
        \item Luminosity decays over time due to degredation of intensities and
            emittances. Mainly from the collisions themselves. Net luminosity
            lifetime is ${\tau_{\lumi} = \SI{14.9}{h}}$.
        \item All of the above: \cite{Bruning:782076}.
        \item Luminosity report by CMS: \cite{CMS-PAS-LUM-17-001}.
    \end{itemize}

    %------------------------------------------------------------------------------%
    \section{Compact Muon Solenoid (\CMS) Experiment}

    \begin{itemize}
        \item CMS 2006 detector TDR: \cite{Bayatian:922757}.
        \item CMS 2007 physics TDR: \cite{Bayatian:942733}.
        \item CMS 2005 computing TDR: \cite{Bayatyan:838359}.
        \item CMS 2000 TriDAS TDR: \cite{Bayatyan:706847}.
        \item CMS 2013 L1 trigger upgrade TDR: \cite{Tapper:1556311}.
        \item CMS 2012 HCAL upgrade TDR: \cite{Mans:1481837}.
        \item Layout and design centered around the magnetic field for measuring
            muon momenta. ${\SI{13}{m}}$ long, ${\SI{5.9}{m}}$ inner diameter,
            ${\SI{3.8}{T}}$ superconducting solenoid.
        \item Segmented into various regions in $\eta$ coverage known as the barrel,
            endcap and forward regions. The magnet leads to the segmentation between
            the inner and outer detectors. Further segmentation is done in each
            subdetector.
        \item Return field saturates ${\SI{1.5}{m}}$ iron, allowing 4 muon stations.
        \item The bore of the magnet contains the inner tracker, electromagnetic
            calorimeter and hadronic calorimeter
        \item ${\SI{21.6}{m}}$ long and ${\SI{14.6}{m}}$ in diameter weighing
            ${\SI{12500}{tons}}$. ECAL is greater than 25 radition lengths. HCAL
            varies between 7--11 interaction lengths.
        \item 20 pileup collisions result in 1000 charged particles every
            ${\SI{25}{ns}}$ bunch crossing. Require high-granularity detectors
            with good time resolution for low occupancy. This results in large
            number of detector channels and millions of synchronised detector
            electronic channels. Components near the beam line receive a large
            flux of particles leading to high radition levels, requiring
            radition-hard detectors and front-end electronics.
    \end{itemize}

    \subsection{Silicon Tracker}

    \begin{itemize}
        \item Silicon technology with reverse-biased p-n junctions. An ionising
            particle (i.e. charged) creates electron-hole pairs which drift in the
            electric field towards a cathode/anode.
        \item ${\SI{5.8}{m}}$ long, ${\SI{2.6}{m}}$ diameter tracking volume
        \item 10 layers of silicon microstrip detectors provide required granularity
        and precision.
    \item Additional 3 layers placed near the interation region to improve
        measurement of the impact parameter of charged particles and position
        of secondary vertices.
    \item Cover region up to ${\aeta=2.5}$.
\end{itemize}

\subsection{\ECAL}

\begin{itemize}
    \item Lead tungstate crystals cover up to ${\aeta=3.0}$.
    \item Lead tungstate have short radiation lengths (${\SI{0.89}{cm}}$) and
        Moliere (${\SI{2.2}{cm}}$) lengths with fast response times and are
        radiation hard. However, they have a low light yield
        (${{30}{\gamma/MeV}}$) requiring photodetectors with intrinsic gain
        operable in the magnetic field. Silicon avalanche photodiodes (require
        stable temperature, goal: ${\SI{0.1}{\celsius}}$) are used in the
        barrel and vacuum phototriodes in the endcaps.
    \item 61200 crystals in the central barrel and 7324 crystals in each of the
        two endcaps.
    \item Scintillation light detected by silicon avalanche photodiodes in the
        barrel region. Vacuum phototriodes in the endcap region.
    \item Preshower system installed in front of the endcap ECAL for
        ${\Ppizero}$ rejection. 2 planes of silicon strip detectors behind
        disks of lead absorber at depths of 2 and 3 radiation lengths.
    \item Barrel covers ${\aeta<1.479}$, ${\SI{0.0174}{rad}}$ in \dphi and
        \deta with a length of ${\SI{230}{mm}}$ for 25.8 radiation lengths.
    \item Endcaps cover ${1.479<\aeta<3.0}$ in an $x$-$y$ grid with a front
        cross section of ${28.6\times\SI{28.6}{mm^2}}$ and length of
        ${\SI{220}{mm}}$ (24.7 radiation lengths).
    \item Resolution parameterised by
        \begin{equation}
            \left( \frac{\sigma}{E} \right)^{2} = \left( \frac{S}{\sqrt{E}} \right)^{2}
            + \left( \frac{N}{E} \right)^{2} + C^2,
        \end{equation}
        where $S$ stochastic term, N the noise and C constant term. Total energy
        resolution ranges between ${1.50--0.35\%}$ for ${10--\SI{250}{GeV}}$.
        Beyond which the constant term dominates.
\end{itemize}

\subsection{\HCAL}

\begin{itemize}
    \item Brass/scintillator sampling hadron calorimeter with coverage up to
        ${\aeta=3.0}$. Scintillation detected by wavelength-shifting fibres
        embedded in the scintillator tiles and channeled to photodetectors
        via clear fibres. Then detected by hybrid photodetectors the provide
        gain and operate in high axial magnetic fields. Outer calorimeter
        in the iron return yoke beyond the magnet covering the barrel region
        adds about three interaction lengths ontop of the 8 provided by the
        inner calorimeter.
    \item Coverage up to ${\aeta=5.0}$ provided by iron/quartz-fibre
        calorimeter. Cerenkov light emitted by the quartz fibres is detected by
        photomultiplers. Ensure full coverage for the measurement of the
        transverse energy of an event.
\end{itemize}

\subsection{Solenoid Magnet}

\begin{itemize}
    \item High-purity aluminium-stabilised conductor and indirect cooling (by
        thermosyphon).
    \item Bending power is designed around measuring muon momenta to a
        resolution of ${\Delta\mom/\mom\approx 10\%}$ at ${\mom = \SI{1}{TeV}}$.
    \item Favourable length/radius ratio chosen to ensure good momentum
        resolution in the forward region.
\end{itemize}

\subsection{Muon Chambers}

\begin{itemize}
    \item Resolution limited by multiple scattering in the central material
        up to $\pt=\SI{200}{GeV}$, when the chamber spatial resolution starts
        to dominate.
    \item Three types of gaseous deetctors.
    \item Barrel region (${\aeta<1.2}$), with low neutron induced background,
        the muon rate is low and residual field is low, drift tube chambers are
        used.
    \item 250 DT chambers in 4 layers. Staggered chambers allow high-\pt muons
        near boundaries to cross at least 3 out of the 4 layers. Each station is
        designed to give $\phi$ precision better and ${\SI{100}{\micro m}}$ in
        position and approximately ${\SI{1}{\milli rad}}$ in direction.
    \item The 2 endcaps, high muon and neutron induced background rate is high,
        and large field, cathode strip chambers are used to cover up to
        ${{\aeta}=2.4}$.
    \item 468 CSCs in 2 endcaps. Trapezoidal shape consisting of 6 gas gaps,
        each gap having radial cathode strips and a plane of anode wires running
        perpendicular to the strips. CSCs are overlapped in phi to avoid gaps
        in muon acceptance. Gas ionization and subsequent electron avalanche
        caused by charged particles produces a charge on the anode and image
        charge on the cathode. Precise precision measurement is made by
        determining the centre-of-gravity of the charge distribution induced on
        the cathode strips. Spatial resolution provided by each chamber from the
        strips is typically ${\SIrange{100}{200}{\micro m}}$ with an angular
        resolution in $\phi$ of the order of ${\SI{10}{\milli rad}}$.
    \item Resistive plate chambers are used in both regions. Operating
        in avalanche mode for good operation at high rates. Good timing
        resolution with coarse position resolution to identify the correct
        bunch crossing. Coverage up to ${\aeta<1.6}$ or ${\aeta<2.4}$.
\end{itemize}

\subsection{Hardware Trigger}

\subsection{Software Trigger}

\subsection{Worldwide \LHC Computing Grid (\WLCG)}

%------------------------------------------------------------------------------%
